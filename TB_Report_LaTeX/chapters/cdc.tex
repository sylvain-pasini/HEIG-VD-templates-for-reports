% +---------------------------------------------------------------+
% indique quel est le document Master à compiler
% TeXstudio
% !TeX master = ../tb_report.tex
%
% TexShop
% !TEX root = ../tb_report.tex

\chapter{Cahier des charges}


\section*{Résumé du problème}
%exemple
\lipsum[110-111]

\subsection*{Problématique}
%exemple
\lipsum[110-110]

\section*{Cahier des charges}
%exemple
\lipsum[1-1]

\subsection*{Objectifs}
%exemple
\todo{redefinir mieux les objectifs}
\lipsum[1-1]

\subsection*{Déroulement}
Le travail commence le ... et se termine le .... Sur les 16 premières semaines, soit du x au y, la charge de travail représente 12h par semaine. Les 6 dernières semaines, soit du x au y, ce travail sera réalisé à plein temps.

Un rendu intermédiaire noté est demandé le x et le rendu final est prévu pour le x à 12h00.

La défense sera organisée entre le x et le y.

\subsection*{Tâches}
%exemple
\lipsum[1-1]

\subsection*{Livrables}
\todo{à contrôler avec mon superviseur}
Les délivrables seront les suivants :
\begin{enumerate}
	\item Une documentation contenant :
	      \begin{itemize}
		      \item Une analyse de marché
		      \item La décision qui découle de l’analyse
		      \item Spécifications
		      \item Les informations du module tel que le fonctionnement et les limitations
		      \item Une planification initiale et finale
		      \item Un mode d’emploi
	      \end{itemize}
	\item Un module remplissant les objectifs défini au point 2.1.
	\item Un software implémentant les améliorations s’il a été possible de les effectuer.
\end{enumerate}

